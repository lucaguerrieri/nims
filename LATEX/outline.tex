\documentclass[12pt]{article}
\usepackage{amssymb}
\usepackage{amsmath}
\usepackage{graphicx}
\usepackage{chicago}

\setlength{\paperwidth}{8.5in} \setlength{\paperheight}{11.0in}
\setlength{\topmargin}{0.0in} \setlength{\headheight}{0.4in}
\setlength{\headsep}{0.0in} \setlength{\textwidth}{7.2in}
\setlength{\textheight}{8.5in} \setlength{\oddsidemargin}{0.0in}
\setlength{\oddsidemargin}{-0.4in}
\setlength{\evensidemargin}{-0.4in}

\renewcommand{\baselinestretch}{1.5}
\renewcommand{\textfraction}{0.33}
\begin{document}

\bigskip

\bigskip

\clearpage

\renewcommand{\baselinestretch}{1.0}

\begin{center}
{\normalsize \thispagestyle{empty} }

{\normalsize \medskip }

{\normalsize {\Large Forcasting Net Interest Margins in a Zero Interest Rate Environment  $^{*}$%
} \medskip }

{\normalsize \bigskip \bigskip }

{\normalsize Valentin Bolotnyy, Rochelle Edge, and Luca Guerrieri }

{\normalsize \bigskip \bigskip }

{\normalsize April 29, 2013 }

{\normalsize \bigskip }
\end{center}

{\normalsize \bigskip }

\abstract{We forecast net interest margins (NIMs) using a simple vector autoregression model that includes a short interest rate and a long interest rate,
and using a term structure modeled after Diebold, et al.. Comparing NIM forecasts produced by each of these models, we find that .... Modeling the zero lower bound as a regime change produces .... Additionally, we find that NIM forecasts aggregated for the top 25 U.S. banks, by assets, are ... compared to firm-specific forecasts which are .... Exploring the effects of shocks
at different points in the term structure, we find ....}

{\normalsize \vspace{3.0cm} }

{\normalsize \noindent \textbf{Keywords}:  \vspace{1cm} }

{\normalsize \noindent \textbf{JEL Classification}:  }

{\normalsize \vspace{2cm} }

\renewcommand{\baselinestretch}{1} \footnotesize \noindent

\textbf{\ Affiliation and contact information}: Valentin Bolotnyy,
Federal Reserve Board, telephone (202) 452-6428, email
valentin.bolotnyy@frb.gov; Rochelle Edge, Federal Reserve Board, telephone (202) 452-2339, email
rochelle.m.edge@frb.gov; Luca Guerrieri, Federal Reserve Board,
telephone (202) 452-2550, email luca.guerrieri@frb.gov.

\vspace{2cm}

{\footnotesize \noindent $^{*}$ The views expressed in this paper are solely
the responsibility of the authors and should not be interpreted as
reflecting the views of the Board of Governors of the Federal Reserve System
or of any other person associated with the Federal Reserve System. We want to thank Michiel De Pooter, Canlin Li, Marcel Priebsch, John Sears, Emre Yoldas... }

\clearpage \renewcommand{\baselinestretch}{1.5} \normalsize

\section{Outline}
1. Introduction: Motivate desire/need to forecast NIMs. Describe (chart) historical NIM behavior for top 25 U.S. banks. Summarize our approaches and findings.
\newline
2. Lit Review: Discuss approaches previously taken in the literature to forecasting NIMs.
\newline
3. Methodology: 

    a) Describe VAR model with 3 month and 10 year yields and how we employ it to forecast NIMs
    
    b) Describe Diebold, et al. approach to modeling yield curve and how we employ it to forecast NIMs
    
    c) Discuss why we chose to focus on these two models
    
    d) Discuss how why we chose to deal with zero lower bound as a regime shift
    
    e) Describe how we test forecasts (Diebold-Mariano and Clark-West)
\newline
4. Results: Discuss forecasts produced by different models and how well they do. Do firm-specific forecasts reveal more than aggregate (top 25) forecasts?
\newline
5. Interest Rate Path Scenarios: Discuss how shocks at different points in the term structure would affect NIMs (at aggregate and individual firm levels?)
\newline
6. Appendix: Discuss results of our effort to forecast NIMs by asset class
\newline
Potential Robustness Checks:

    a) Instead of 3-month and 6-month Svensson model interest rates, use implied rates posted by NY Fed.

\end{document}
