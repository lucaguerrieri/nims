\documentclass[12pt]{article}
\usepackage{amssymb}
\usepackage{amsmath}
\usepackage{graphicx}
\usepackage{chicago}
\usepackage{rotating}

\setlength{\paperwidth}{8.5in} \setlength{\paperheight}{11.0in}
\setlength{\topmargin}{0.0in} \setlength{\headheight}{0.4in}
\setlength{\headsep}{0.0in} \setlength{\textwidth}{7.2in}
\setlength{\textheight}{8.5in} \setlength{\oddsidemargin}{0.0in}
\setlength{\oddsidemargin}{-0.4in}
\setlength{\evensidemargin}{-0.4in}

\newtheorem{theorem}{Theorem}
\newtheorem{acknowledgement}[theorem]{Acknowledgement}
\newtheorem{algorithm}[theorem]{Algorithm}
\newtheorem{axiom}[theorem]{Axiom}
\newtheorem{case}[theorem]{Case}
\newtheorem{claim}[theorem]{Claim}
\newtheorem{conclusion}[theorem]{Conclusion}
\newtheorem{condition}[theorem]{Condition}
\newtheorem{conjecture}[theorem]{Conjecture}
\newtheorem{corollary}[theorem]{Corollary}
\newtheorem{criterion}[theorem]{Criterion}
\newtheorem{definition}[theorem]{Definition}
\newtheorem{example}[theorem]{Example}
\newtheorem{exercise}[theorem]{Exercise}
\newtheorem{lemma}[theorem]{Lemma}
\newtheorem{notation}[theorem]{Notation}
\newtheorem{problem}[theorem]{Problem}
\newtheorem{proposition}[theorem]{Proposition}
\newtheorem{remark}[theorem]{Remark}
\newtheorem{solution}[theorem]{Solution}
\newtheorem{summary}[theorem]{Summary}
\newenvironment{proof}[1][Proof]{\noindent\textbf{#1.} }{\ \rule{0.5em}{0.5em}}
\input{tcilatex}

\renewcommand{\baselinestretch}{1.5}
\renewcommand{\textfraction}{0.33}
\begin{document}

\bigskip

\bigskip

\clearpage

\renewcommand{\baselinestretch}{1.0}

\begin{center}
{\normalsize \thispagestyle{empty} }

{\normalsize \medskip }

{\normalsize {\Large Forecasting Net Interest Margins of Banks$^{*}$%
} \medskip }

{\normalsize \bigskip \bigskip }

{\normalsize Valentin Bolotnyy, Rochelle Edge, and Luca Guerrieri }

{\normalsize \bigskip \bigskip }

{\normalsize April 29, 2013 }

{\normalsize \bigskip }
\end{center}

{\normalsize \bigskip }

\abstract{}

{\normalsize \vspace{3.0cm} }

{\normalsize \noindent \textbf{Keywords}:  \vspace{1cm} }

{\normalsize \noindent \textbf{JEL Classification}:  }

{\normalsize \vspace{2cm} }

\renewcommand{\baselinestretch}{1} \footnotesize \noindent

\textbf{\ Affiliation and contact information}: Valentin Bolotnyy,
Federal Reserve Board, telephone (202) 452-6428, email
valentin.bolotnyy@frb.gov; Rochelle Edge, Federal Reserve Board, telephone (202) 452-2339, email
rochelle.m.edge@frb.gov; Luca Guerrieri, Federal Reserve Board,
telephone (202) 452-2550, email luca.guerrieri@frb.gov.

\vspace{2cm}

{\footnotesize \noindent $^{*}$ The views expressed in this paper are solely
the responsibility of the authors and should not be interpreted as
reflecting the views of the Board of Governors of the Federal Reserve System
or of any other person associated with the Federal Reserve System.}

\clearpage \renewcommand{\baselinestretch}{1.5} \normalsize

\section{Introduction}

Regular bank stress testing (and the capital planning that it implies) is one of the three complementary reforms that has been made to the capital regulatory regime made in response to the shortcomings of the pre-crisis system.   While stress testing can take many forms - such as, being based on a small number of scenarios, being based on a large number of scenarios, or even being undertaken in reverse form to achieve a certain adverse outcome - the form of stress testing mandated under the Dodd-Frank Act for U.S. bank regulators and the form being used by the Federal Reserve for its annual Comprehensive Capital Analysis and Review (CCAR) is one based around small number of scenarios.  This approach involves specifying a small set of macroeconomic and financial scenarios that represents stressful conditions over the time horizon of the stress test and then projecting, for each BHC in the stress test, its losses, income, and path of pro-forma capital based on the scenario.  Clearly, therefore, there are at least two important parts to conducting bank stress tests:  Formulating the macroeconomic and financial scenarios that a priori would seem stressful to banks and developing models that are able to translate such stressful conditions into losses, income, and pro-forma capital.  This paper considers an aspect of the latter of these two issues.
	In developing models to translate stressful macroeconomic and financial conditions into bank income- and balance-sheet variables it is useful to consider the types of developments, and for which particular variables, that are likely to be stressful to banks.  This is because adverse developments in these variables are very likely to be featured in one of the specified stressful scenarios.  Moreover, it is also desirable that the models being used for stress testing are able to translate developments in these variables well into the key income- or balance-sheet variable that the model is designed to project.  In this paper, we examine this property for models of bank net interest margins (NIMs) defined as the ratio of net interest income (NII) to interest earning assets, where NII is the difference between a bank's interest income and interest expenses.
	Our motivation for focusing on models of NIMs is based on a number of considerations.  First,  bank losses that arise from banks' having to pay out more in interest of their liabilities than they are earning from their assets should not be overlooked as an important source of risk to the banking and broader financial sector.  To be sure, this form of losses was much less prominent in the most recent crisis relative to the form of bank losses that stem from loan defaults.  But, there are ample examples of other earlier banking crises in which adverse developments in NIMs have played a significant role.  The most familiar of these from the U.S. perspective is the Savings and Loans (S\&L) crisis of the late 1980s and early 1990s in which - as a result of the Volcker disinflation - short-term interest rates rose above long-term interest rates, interest expenses rose above interest income, NII and NIMs turned negative (in the thrift sector), and sizable losses and a substantial number of bank failures resulted.  Likewise, the three Nordic banking crises of the late 1980s and early 1990s (for Finland, Sweden, and Norway) - which represent three out of Reinhart's and Rogoff's  (2009) "big five" banking crises - were associated with high short-term interest rates that turned NII and NIMs negative and lead to sizable bank losses and failures.   Additionally, the U.K.'s secondary banking crisis of the early 1970s was associated with a sudden increase in short-term interest rates that abruptly drove up banks' costs of funds so too leading to losses and failures.  To be sure, large increases in short-term rates and declines in net interest income were not the only reason for any of these crises but the role of short-term interest rates - and, in particular, their elevated level relative to long-term interest rates was material and in most cases equivalent in importance to the other causes of the crisis.
	Another motivation for our focus on models of NIM stems from the recent lackluster pace of bank profit growth.   Bank net income growth in recent years has been largely supported by unsustainable cost cutting and reductions in provisioning, rather than from traditional sources of bank revenue-generation like NII.  NII growth, for example, (in both nominal and real terms) has been well below average in recent years while NIMs (NII relative to interest earning assets) have also been low.   Given their already low level, any adverse development that puts further downward pressure on NIMs could be very stressful for banks and, as such, it is important that the economic developments (likely involving interest rates) that would drive these outcomes are well-modeled.
Figure 1 - Net interest income growth (LHS) and net interest margins (RHS)

	Our final motivation for examining NIMs is that despite the importance of NIMs for the financial condition of banks, the modeling of NIMs has received much less attention in the literature relative to modeling of bank financial-statement variables related to credit losses.  To be sure, the recent increase in the importance of stress testing has led to the development of a number of models that link NIMs to macroecnomic variables - such as, short-term Treasury rates and the slope of the Treasury yield curve - where Kovner et al. (2011) and Covas et al. (2012) are notable examples.   However, this research has given relatively little attention to the conditional forecasting properties of these models, which is important if these models are ultimately to be used to project bank losses, income, and pro-forma given macroeconomic scenarios.
	In this paper we consider a range of different models of NIMs, developed with varying degrees of modeling complexity, and examine how well these models explain NIMs.  An important emphasis in examining how well these models explain NIMs is pseudo-out-of-sample forecast performance.  We adopt this approach for the simple reason that the evaluation of models based on their in-sample forecast performance, frequently leads to the econometrician over-fitting the model within the sample so as to maximize in-sample forecast performance.  And this leads to the model performing very poorly once it starts to be used beyond the estimation sample.  Model evaluation based on pseudo-out-of-sample forecast performance indemnifies the econometrician somewhat from this tendency.  We also try to limit the tendency to fine-tune the model to obtain better performing (now pseudo-out-of-sample) forecasts by remaining fairly restrained in terms of choosing additional variables to include in the model.  For the most part we use only the most obvious variables in our models; specifically, those that we can make a fairly strong theoretical justification for and those that appear frequently in other models of NIMs.
	We have already noted a couple of papers that attempt to model NIMs, but there are several additional ones.  OF RECENT PAPERS SKANDER AND LISA.  NOTE TWO PARTS TO THE LITERATURE OLD AND NEW AND BANKING AND MACRO.


\section{Models}
\indent 1. Forecast Combination:

2. Dynamic Factor Model, with restrictions:

Transition equation, which governs the dynamics of the state vector:

$\begin{pmatrix} {L_t - \mu_L} \\ {S_t - \mu_S} \\ {C_t - \mu_C} \end{pmatrix}$ =
$\begin{pmatrix} a_{11} & a_{12} & a_{13} \\ a_{21} & a_{22} & a_{23} \\ a_{31} & a_{32} & a_{33} \end{pmatrix}$
$\begin{pmatrix} {L_t - \mu_L} \\ {S_t - \mu_S} \\ {C_t - \mu_C} \end{pmatrix}$ +
$\begin{pmatrix} {\eta_t(L)} \\ {\eta_t(S)} \\ {\eta_t(C)} \end{pmatrix}$

Measurement equation, which relates a set of $N$ yields (in our case, 12), to the three unobservable factors and NIMs:

%$\begin{pmatrix} y_t(\tau_1) \\ y_t(\tau_2) \\ \vdots \\ y_t(\tau_$N$) \end{pmatrix}$

3. Dynamic Factor Model with 2-Step Regression:
	
	a. Multivariate (3 Factors added all together to NIM equation)

	b. Forecast Combination (3 Factors added one at a time and combined)

	c. With Poor-Man's Factors

4. VAR:
	
	a. Poor-Man's Factors:

	b. Smoothed Factors:

5. No-Change Forecast:

This forecast uses $y_t$ as the forecast for $y_{t+1}$.


\section{Comparing Forecast Performance}

\clearpage

\begin{sidewaystable}
\center
\caption{RMSE for shortened sample}
\begin{tabular}{|l|c|c|c|c|c|c|c|c|c|c|}
\hline
&Step 1 &Step 2 &Step 3 &Step 4 &Step 5 &Step 6 &Step 7 &Step 8 &Step 9 &Step 10\\
\hline
Forecast Combination of Yields        &0.107&0.133&0.148&0.168&0.187&0.202&0.213&0.209&0.216&0.223\\
DFM                                   &0.122&0.148&0.169&0.212&0.247&0.278&0.316&0.340&0.373&0.404\\
DFM + 2nd Step Regression             &0.118&0.167&0.209&0.260&0.308&0.356&0.398&0.433&0.474&0.511\\
DFM + Forecast Combination            &0.112&0.159&0.199&0.247&0.300&0.351&0.399&0.438&0.489&0.535\\
Forecast Combination of Simple Factors&0.118&0.169&0.216&0.273&0.336&0.399&0.456&0.509&0.567&0.622\\
VAR on DF                             &0.173&0.234&0.314&0.379&0.412&0.436&0.436&0.476&0.523&0.571\\
VAR on Simple Factors                 &0.166&0.209&0.263&0.302&0.342&0.375&0.380&0.430&0.451&0.499\\
No-Change Forecast                    &0.107&0.138&0.154&0.181&0.206&0.228&0.244&0.248&0.271&0.293\\
\hline
\end{tabular}
\end{sidewaystable}



\clearpage
\begin{sidewaystable}
\caption{RMSEs over full sample}
\center
\begin{tabular}{|l|c|c|c|c|c|c|c|c|c|c|}
\hline
&Step 1 &Step 2 &Step 3 &Step 4 &Step 5 &Step 6 &Step 7 &Step 8 &Step 9 &Step 10\\
\hline
Forecast Combination of Yields        &0.105&0.140&0.164&0.191&0.218&0.244&0.265&0.281&0.305&0.327\\
DFM + 2nd Step Regression             &0.128&0.190&0.244&0.299&0.350&0.397&0.438&0.475&0.514&0.549\\
DFM + Forecast Combination            &0.112&0.158&0.197&0.240&0.290&0.337&0.373&0.407&0.446&0.480\\
Forecast Combination of Simple Factors&0.116&0.174&0.227&0.284&0.347&0.409&0.459&0.508&0.554&0.595\\
VAR on DF                             &0.156&0.220&0.245&0.218&0.187&0.223&0.260&0.312&0.382&0.431\\
VAR on Simple Factors                 &0.148&0.196&0.243&0.275&0.283&0.296&0.319&0.377&0.456&0.530\\
No-Change Forecast                    &0.107&0.138&0.154&0.181&0.206&0.228&0.244&0.248&0.271&0.293\\
\hline
\end{tabular}
\end{sidewaystable}



\clearpage
\begin{sidewaystable}
\caption{RMSEs for models that include the share of asset of the shadow banking sector}
\center
\begin{tabular}{|l|c|c|c|c|c|c|c|c|c|c|}
\hline
&Step 1 &Step 2 &Step 3 &Step 4 &Step 5 &Step 6 &Step 7 &Step 8 &Step 9 &Step 10\\
\hline
Forecast Combination of Yields        &0.107&0.138&0.157&0.182&0.207&0.232&0.247&0.253&0.266&0.278\\
DFM + 2nd Step Regression             &0.121&0.170&0.212&0.262&0.312&0.360&0.403&0.436&0.479&0.516\\
DFM + Forecast Combination            &0.103&0.135&0.156&0.183&0.207&0.232&0.247&0.253&0.266&0.278\\
Forecast Combination of Simple Factors&0.106&0.136&0.156&0.183&0.207&0.232&0.247&0.253&0.266&0.278\\
VAR on DF                             &0.357&0.425&0.469&0.498&0.569&0.645&0.731&0.781&0.821&0.847\\
VAR on Simple Factors                 &0.338&0.355&0.395&0.356&0.372&0.393&0.459&0.491&0.545&0.572\\
No-Change Forecast                    &0.107&0.138&0.154&0.181&0.206&0.228&0.244&0.248&0.271&0.293\\
\hline
\end{tabular}
\end{sidewaystable}



\end{document}
